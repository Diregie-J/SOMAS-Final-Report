\chapter{Inter Island Trade Organisation (IITO)}

The role of IITO is to organise communication between islands to facilitate inter-island agreements and the giving and receiving of gifts.  

\section{Inter-Island Agreements}  
\label{sec:IITO:inter_island_agreements}  

% Inter-island agreements:
%     - Foraging group formation
%     - Group foraging location
%     - Intended common pool donation
%     - Want to provide islands with enough freedom of communication to allow for interesting simulation
%     - Limited by complexity of agent - too much freedom == not enough time for implementation / overly complex strategies 

Islands being able to reach agreements independently from the main governing body allows for more complex island behaviour. With the implemented system, islands are able to:

\begin{itemize}
    \item Form a group for collective foraging.
    \item Decide where their foraging group will forage.
    \item Inform other islands of the amount they intend to to donate to the common pool.
\end{itemize}

Allowing islands to decide between each other on the groups that they will forage in and where that group will forage means that they are able to work with islands that they trust while avoiding islands that have misbehaved or broken rules in the past. Islands may also want to avoid foraging with other islands that have contributed less than expected in the past if they suspect that they may have been deliberately freeloading.

Allowing islands to broadcast how much they plan to donate to the common pool lets them declare to the other islands if they plan to donate more than the specified tax amount put forward by the president as a form of virtue signalling. This may help an island redeem itself if it had previously lost trust. An island is able to lie about the amount it will donate if it is trying to maliciously gain favour but this can be caught by the Judge if they choose to inspect it, which may result in punishment.

\section{Gifting}  
\label{sec:IITO:gifting}  

% Gifting:
%     - Struggling islands are able to request gifts from other islands without breaching rules and taking too much from common pool
%     - Allows islands to build faith among each other 
%     - Overall island survival may be in best interest of individual islands

If an island is struggling and requires resources to survive it may ask the President for more from the common pool but the President may not accept this request. The struggling island is still able to take money from the common pool if rejected but law abiding islands would probably want to avoid this. To give them another option they are able to request resources in the form of a gift from the other islands. Islands may accept this request if they view the survival of the requesting islands as beneficial or if they wish to improve their standing with the requesting islands. When requesting, giving or receiving a gift, the island is able to specify a reason for their action, which gives islands more information about the transaction. For example a requesting island may want to specify that they will move to a critical state in the next round if they do not receive the gift or an offering island may want to specify that the gift is meant as a reward for successful disaster forecasting.

Islands are also able to offer gifts without a request being made, meaning they can reach out to other islands if they want to boost their popularity.