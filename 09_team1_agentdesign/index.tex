\chapter{Team 1 Agent Design}

\section{Core Idea}
Team 1 Agent was designed around the idea that the agent wants the whole archipelago to survive. However, the agent does have different configurations that allows it to be more malicious than intended such that interesting simulations may occur.

\section{Opinions on Islands}
For an agent to become self-organising, the agent must determine certain actions given a condition without external input. For this to be possible, the agent must have defined conditions which Team 1 calls 'emotional state', and hold an opinion about other islands. This will form a basis for the agent to decide on an action.

Initially, the opinion on all existing islands are neutral. Overtime through IITO and IIGO, the opinions on islands will change. This will affect the outcome of IITO and IIGO results. As a note, positive values correspond to positive opinions while negative values correspond to negative opinions.

An agent's emotional state can change depending on the current resources. Below is an example of how an agent's emotional state can be formed:
\begin{itemize}
    \item If current resources > 10 * living cost, the agent is happy.
    \item If 3 * living cost < current resources < 10 * living cost, the agent is anxious.
    \item If the current resources < 3 * living cost, the agent is desperate.
\end{itemize}
The upper and lower bounds can be changed before a game starts. 

\section{IITO Gifts}
During IITO, the agent's opinion of other island is affected. For every gift received, the agent's opinion of the gifter increases. However, the agent's opinion of an island can decrease if that island promised a gift and was not able to fulfil it. 

When team 1 agent receives a request for gifts, the agent will decide how much to offer depending on the agent's current emotional state and the opinion of that island. If the agent is desperate, then the agent will refuse all gift requests therefore offering a gift of 0. If the agent is anxious, then gift offers will be a proportion of the current resources that the agent is willing to give away. If the agent is happy, then the agent will freely offer resources that satisfy the incoming gift request amount.

Moreover, if the agent's opinion of an island is very high, the agent can decide to give gifts disregarding the agent's own anxiety. On the other hand, if an opinion of an island is very low, the agent can decide to refuse to send a gift even though the agent is happy. 

For increase survivability, team 1 agent will accept any gift offers that it receives. 

\section{IIFO Disaster}
Disasters can happen deterministically or stochastically (mentioned in detail in Chapter~\ref{sec: Disaster} for more information). For an agent, it is important to determine when a disaster occurs so that as much disaster damage is mitigated using the common pool. 
% (see Chapter \ref{sec:IIFO:ltCRD} for more information). 

When the game starts, the disaster prediction made by the agent is random. This prediction always has a confidence value of 0. As more disasters occur, a history of disasters is built up. Using this history, the mean disaster position x, position y, magnitude and occurrence is calculated. A confidence value is calculated along with the mean disaster metrics. 

% Add a footnote on website?  https://www.mathsisfun.com/data/confidence-interval.html
The confidence value is calculated by finding the ratio between margin of error and the mean value. The smaller the margin of error, the more confident the agent is. Therefore, a difference between the mean value and the margin of error must also be calculated. To begin with, the agent uses the confidence interval equation (where $s =$ standard deviation, $n = $ size of array, $Z = $ confidence interval) to calculate the margin of error:
\begin{equation}
    \label{eq: Team1MarginOfError}
    \text{Error} = Z \dfrac{s}{\sqrt{n}} 
\end{equation}
Using the difference between the mean value ($\bar{x}$) and the margin of error along with finding out the ratio of this result will provide the agent with the confidence value.
\begin{equation}
    \text{Confidence Value} = (\bar{x} - \text{Error}) \times \dfrac{1}{\bar{x}}
\end{equation}

Sharing and obtaining other disaster information to and from other islands respectively can increase the survivability of the archipelago. As more disaster prediction is shared, a network of trust between team 1 agent and other island is built. This trust value is primarily based upon the absolute value of the islands prediction on the day the disaster happened. However, if an island shares a disaster prediction with the estimated disaster day to be random or changing erratically each turn, then team 1 agent will begin to distrust that island. 

\section{Foraging}
Multiple foraging strategies were developed: 
